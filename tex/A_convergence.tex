% !TeX root = ../dissertation.tex


\begin{figure}[ht]
    \centering
    \begin{tikzcd}[row sep=huge, column sep=huge]
        \text{unif}
            \arrow[r, ForestGreen]
            \arrow[d, ForestGreen]
        &  L^∞
            \arrow[r, ForestGreen]
            \arrow[d, ForestGreen]
        &  L^2
            \arrow[r, ForestGreen]
            \arrow[dl, densely dotted, red, swap, near start, "\text{†}" description]
        &  L^1
            \arrow[r, ForestGreen]
        &  \Pr
            \arrow[r, ForestGreen]
            \arrow[dlll, bend left, dash dot, magenta, "∀ \text{ subseq } ∃ \text{ subsubseq }"]
            \arrow[l, bend right, densely dotted, red, swap, "\text{UI}"]
        &  d
            \arrow[l, bend right, swap, densely dotted, red, "\text{const}"]  \\
        \text{sure}
            \arrow[r, ForestGreen]
        &  \text{a.s.}
            \arrow[urrr, ForestGreen]
            \arrow[urr, densely dotted, red, near end, "\text{LDCT}" description]
    \end{tikzcd}
    \caption[Modes of convergence]{Relationship between modes of convergence. Green solid arrows signify automatic implication. Red dotted arrows require additional conditions mentioned on the arrow. Magenta dot-dashed lines cover all other cases.}
    \label{fig:convergence_modes}
\end{figure}
